\documentclass[•]{article}
\usepackage{blindtext}
\usepackage[T1]{fontenc}
\usepackage[utf8]{inputenc}
\usepackage{algorithm}
\usepackage[noend]{algpseudocode}
\usepackage{amsmath}

\begin{document}
\title{Perception and localization}
\author{Matteo Baiguera}
\maketitle


\date{\today}

\section{Introduction}
Perception: detect the object, classify the object, estimate the object's trajectory.\\
The assumed sensor-suite is the following 
\begin{center}
\begin{tabular}{c|c|c|}
\hline
				   & Sensor(s)& computing platform\\
\hline
\textbf{Perception}& LiDAR, Camera&\\
\hline
					& Ultrasonic&\\
\hline
					& Event camera&\\
\hline
\hline
\textbf{Navigation}& IMU,GPS&\\
\hline
\end{tabular}
\end{center}

%Therefore sensors are splitted into two main classes
%\begin{itemize}
%\item{\textbf{•}}
%\item{\textbf{•}}
%\end{itemize}


\section{Extended Kalman Filter}

The purpose of extended kalman filter in AD problem is to track objects.\\

The assumed dynamic for all object is the following:
\begin{align*}
\vec{x}_{k+1} &= A_k \vec{x}_k + B_k \vec{u}_k  \\
\vec{x}_{k} &= C_k \vec{z}_k
\end{align*} 

For each tracked object is assumed a state ($\vec{x} $) and an observation ($\vec{z} $).\\
\begin{align*}
\vec{x}_{k+1} &= A_k \vec{x}_k + B_k \vec{u}_k  \\
\vec{x}_{k} &= C_k \vec{z}_k
\end{align*} 

The object state is defined as: 
\begin{align}
    \vec{x} &= \begin{bmatrix}
           	    x  \\
           		y \\
               v_x \\
          	   v_y\\
         \end{bmatrix}
\end{align}

and 


The object's motions is assumed to be a \textsl{"random walker"} process, such that:
\begin{align}
   A_k &= \begin{pmatrix}
           	    1 & 0    & \Delta t & 0  \\
           		0 & 1    & 0        & \Delta t \\
                0 & 0    &        1 & 0 \\
          	    0 & 0    & 0       & 1 \\
         \end{pmatrix}
\end{align}
\begin{algorithm}[H]
\caption{Put your caption here}
\begin{algorithmic}[1]

    \For {$o_j \in {o1,o2,...oN}$} \Comment{track all objects}
%    \If{$condition = True$}
%        \State Do this
%        \If{$Condition \geq 1$}
%        \State Do that
%        \ElsIf{$Condition \neq 5$}
%        \State Do another
%        \State Do that as well
%        \Else
%        \State Do otherwise
%        \EndIf
%    \EndIf
 	\State Initialization: $ $
	\State Update Step: $ $
	\State Predict Step: $ $
    \EndFor
    \While{$something \not= 0$}  \Comment{put some comments here}
        \State $var1 \leftarrow var2$  \Comment{another comment}
        \State $var3 \leftarrow var4$
    \EndWhile  \label{roy's loop}
\end{algorithmic}
\end{algorithm}


\section{Particle Filter}
Vehicle ego-localization via particle filter

\end{document}


